\documentclass[a4paper]{report}

%uncomment to see the references
%\usepackage{showkeys}
\usepackage[T2A]{fontenc}
\usepackage[utf8]{inputenc}

\usepackage{algorithm}
\usepackage{algorithmic}
\usepackage[english,russian]{babel}

%\usepackage[backend=biber,sorting=none,sortcites=true,bibstyle=sty/gost71,maxnames=99,citestyle=numeric-comp,babel=other]{biblatex}
\usepackage[backend=biber,sorting=none,sortcites=true,maxnames=99,citestyle=numeric-comp,babel=other]{biblatex}
%\usepackage[backend=biber]{biblatex}

\defbibenvironment{bibliography}
  {\list
     {\printfield[labelnumberwidth]{labelnumber}.}
     {\setlength{\labelwidth}{2\labelnumberwidth}%
      \setlength{\leftmargin}{\labelwidth}%
      \setlength{\labelsep}{\biblabelsep}%
      \addtolength{\leftmargin}{\labelsep}%
      \setlength{\itemsep}{\bibitemsep}%
      \setlength{\parsep}{\bibparsep}}%
      \renewcommand*{\makelabel}[1]{\hss##1}}
  {\endlist}
  {\item}

\usepackage{csquotes}
%\usepackage{expdlist}
%\usepackage[nottoc,notbib]{tocbibind}
\usepackage[pdftex]{graphicx}
\graphicspath{{pic/}}
\usepackage{amsmath}
\usepackage{amssymb}
\usepackage{amsthm}
\usepackage{amsfonts}
\usepackage{amsxtra}
\usepackage{sty/dbl12}
%\usepackage{srcltx}
\usepackage{epsfig}
%\usepackage{verbatim}
\usepackage{sty/rac}
\usepackage{listings}
\usepackage[singlelinecheck=false]{caption}

\usepackage{xcolor, colortbl}
\definecolor{light-gray}{RGB}{230,230,230}

%%%%%%%%%%%%%%%%%%%%%%%%%%%%%%%%%%%%%%%%%%%%%%%%%%%%%%%%%%%%%%%%%%%%%%%%%%%%%%

\captionsetup[figure]{justification=centering,   position=bottom, skip=0pt}
\captionsetup[table] {justification=raggedright, position=top,    skip=0pt}

% Redefine margins and other page formatting

\setlength{\oddsidemargin}{0.5in}

% Various theorem environments. All of the following have the same numbering
% system as theorem.

\theoremstyle{plain}
\newtheorem{theorem}{Теорема}
\newtheorem{prop}[theorem]{Утверждение}
\newtheorem{corollary}[theorem]{Следствие}
\newtheorem{lemma}[theorem]{Лемма}
\newtheorem{question}[theorem]{Вопрос}
\newtheorem{conjecture}[theorem]{Гипотеза}
\newtheorem{assumption}[theorem]{Предположение}

\theoremstyle{definition}
\newtheorem{definition}[theorem]{Определение}
\newtheorem{notation}[theorem]{Обозначение}
\newtheorem{condition}[theorem]{Условие}
\newtheorem{example}[theorem]{Пример}
%\newtheorem{algorithm}[theorem]{Алгоритм}
\floatname{algorithm}{Листинг}
\renewcommand{\algorithmicrequire}{\textbf{Вход:}}

%\newtheorem{introduction}[theorem]{Introduction}

\renewcommand{\proof}{\\\textbf{Доказательство.}~}

\def\startprog{\begin{lstlisting}[language=Java,basicstyle=\normalsize\ttfamily]}

%\theoremstyle{remark}
%\newtheorem{remark}[theorem]{Remark}
%\include{header}
%%%%%%%%%%%%%%%%%%%%%%%%%%%%%%%%%%%%%%%%%%%%%%%%%%%%%%%%%%%%%%%%%%%%%%%%%%%%%%%

\numberwithin{theorem}{chapter}        % Numbers theorems "x.y" where x
                                        % is the section number, y is the
                                        % theorem number

%\renewcommand{\thetheorem}{\arabic{chapter}.\arabic{theorem}}

%\makeatletter                          % This sequence of commands will
%\let\c@equation\c@theorem              % incorporate equation numbering
%\makeatother                           % into the theorem numbering scheme

%\renewcommand{\theenumi}{(\roman{enumi})}

%%%%%%%%%%%%%%%%%%%%%%%%%%%%%%%%%%%%%%%%%%%%%%%%%%%%%%%%%%%%%%%%%%%%%%%%%%%%%%

\binoppenalty=10000
\relpenalty=10000

\addbibresource{main.bib}
%\bibliography{main}

\begin{document}

% Begin the front matter as required by Rackham dissertation guidelines
\initializefrontsections

\pagestyle{title}

\begin{center}
Санкт-Петербургский национальный исследовательский университет \\ информационных технологий, механики и оптики

\vspace{2cm}

Факультет информационных технологий и программирования

Кафедра компьютерных технологий

\vspace{3cm}

{\Large Якупов Илья Юрьевич}

\vspace{2cm}

\vbox{\LARGE\bfseries
Разработка структуры данных для выполнения \\
инкрементальной недоминирующей сортировки
}

\vspace{4cm}

{\Large Научный руководитель: к.т.н. А.~С.~Станкевич}

\vspace{6cm}

Санкт-Петербург\\ 2015
\end{center}

\newpage

\setcounter{page}{3}
\pagestyle{plain}

\tableofcontents
%\listoffigures

% Chapters
\startthechapters
\startprefacepage

Оптимизацией в математике называется задача нахождения экстремума (минимума или максимума) целевой функции в некоторой области конечномерного векторного пространства, ограниченной набором линейных и/или нелинейных равенств и/или неравенств \cite{wiki_opt}.

В процессе проектирования ставится обычно задача определения наилучших, в некотором смысле, структуры или значений параметров объектов. Такая задача называется оптимизационной. Если оптимизация связана с расчётом оптимальных значений параметров при заданной структуре объекта, то она называется параметрической оптимизацией. Задача выбора оптимальной структуры является структурной оптимизацией.

Стандартная математическая задача оптимизации формулируется таким образом. Среди элементов x, образующих множество X, найти такой элемент x*, который доставляет минимальное значение f(x*) заданной функции f(X).

Многокритериальная оптимизация или программирование — это процесс одновременной оптимизации двух или более конфликтующих целевых функций в заданной области определения.

\printbibliography

%\startappendices
%\input{parts/appendix1.tex}

\end{document}