\chapter{Обзор предметной области}
\label{chapter1}

\section{Эволюционные алгоритмы}
\label{evo_algs}

\subsection{Основные принципы}
\textit{Эволюционные алгоритмы} применяются для решения задач оптимизации и используют механизмы, 
основанные на принципах природной эволюции.\cite{petrova_evo1} 

Эволюционные алгоритмы работают с множеством \textit{особей}-кандидатов на оптимальное решение. 
Как правило, особи представляются в виде точек в $K$-мерном пространстве, где $K$ - количество 
оптимизируемых критериев особи. Целью работы эволюционного алгоритма является нахождение наиболее 
приспособленной особи.

Функция, позволяющая оценить приспособленность особи, называется \textit{функцией приспособленности}.
В задачах многокритериальной оптимизации требуется рассматривать несколько функций приспособленности.

Каждая итерация эволюционного алгоритма работает с \textit{поколением} особей. Особи следующего 
поколения генерируются путем применения к особям текущего поколения операторов скрещивания, 
мутации и отбора.

В результате работы каждой итерации эволюционного алгоритма формируется множество особей следующего 
поколения, которое будет оптимизироваться следующей итерацией эволюционного алгоритма. Как правило, 
размер поколения фиксирован, поэтому в конце каждой итерации наименее приспособленные особи 
отбрасываются.

Эволюционный алгоритм завершает работу, как только достигнуто хотя бы одно из условий останова.
Как правило, в качестве критериев останова используется число итераций, либо значения функций
приспособленности.

\subsection{Популярные реализации}
\subsubsection{NSGA}
Алгоритм \textit{NSGA (Nondominated Sorting Genetic Algorithm)} является одним из первых алгоритмов
многокритериальной оптимизации. \cite{deb_nsga2}

Данный алгоритм на каждой итерации выполняет недоминирующую сортировку за $O(KN^3)$.
Помимо значений функции приспособленности, в качестве критерия перехода особи в следующую популяцию
используется \textit{фенотипическая дистанция}: если дистанция до какой-либо особи из следующего 
поколения меньше, чем задано параметром, то текущая особь в следующее поколение не добавляется. 
\cite{nsga1}

У данного алгоритма есть следующие недостатки: \cite{deb_nsga2}
\begin{enumerate}
\item Высокая вычислительная сложность недоминирующей сортировки;
\item Необходимость подбора параметра, определяющего минимально допустимое фенотипическое
	расстояние между особями;
\item Отсутствие элитизма.
\end{enumerate}

\subsubsection{Классический NSGA-II}
Алгоритм NSGA-II призван исправить основные недостатки алгоритма NSGA: \cite{deb_nsga2}
\begin{enumerate}
\item Сортировка выполняется за $O(KN^2)$;
\item При формировании нового поколения используются лучшие особи как из модифицированной
	(т. е. полученной в результате мутации и скрещивания особей текущего поколения) 
	популяции, так и из исходной;
\item При формировании нового поколения из комбинированной популяции отбрасываются особи
	с наибольшим рангом. Если в популяции остается только часть особей с некоторым рангом,
	выбираются особи с наибольшей \textit{crowding distance} - средней длиной ребра кубоида,
	образованного соседними особями (для крайних особей этот показатель равен $\inf$).
\end{enumerate}

Несмотря на это, у данного алгоритма есть недостаток, свойственный всем \textit{классическим}
(т. е. оперирующим поколениями особей) алгоритмам - невозможность эффективного распараллеливания.
Перед тем, как пересчитать ранг особи, алгоритм должен завершить расчет функций приспособленности 
всех особей. Данную проблему призваны решить \textit{инкрементальные (steady-state}) алгоритмы.
\cite{max_me_ss_nsga2}

\subsubsection{Инкрементальный NSGA-II}
TODO
