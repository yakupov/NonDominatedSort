\chapter{Обзор предметной области}
\label{chapter1}

\section{Эволюционные алгоритмы}
\label{evo_algs}

\subsection{Основные принципы}
\textit{Эволюционные алгоритмы} применяются для решения задач оптимизации и используют механизмы, 
основанные на принципах природной эволюции.\cite{petrova_evo1} 

Эволюционные алгоритмы работают с множеством \textit{особей}-кандидатов на оптимальное решение. 
Как правило, особи представляются в виде точек в $K$-мерном пространстве, где $K$ - количество 
оптимизируемых критериев особи. Целью работы эволюционного алгоритма является нахождение наиболее 
приспособленной особи.

Функция, позволяющая оценить приспособленность особи, называется \textit{функцией приспособленности}.
В задачах многокритериальной оптимизации требуется рассматривать несколько функций приспособленности.

Каждая итерация эволюционного алгоритма работает с \textit{поколением} особей. Особи следующего 
поколения генерируются путем применения к особям текущего поколения операторов скрещивания, 
мутации и отбора.

В результате работы каждой итерации эволюционного алгоритма формируется множество особей следующего 
поколения, которое будет оптимизироваться следующей итерацией эволюционного алгоритма. Как правило, 
размер поколения фиксирован, поэтому в конце каждой итерации наименее приспособленные особи 
отбрасываются.

Эволюционный алгоритм завершает работу, как только достигнуто хотя бы одно из условий останова.
Как правило, в качестве критериев останова используется число итераций, либо значения функций
приспособленности.

\subsection{Популярные реализации}
\subsubsection{NSGA-II}
TODO