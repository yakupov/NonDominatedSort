\chapter{Обзор предметной области}
\label{chapter1}

\section{Эволюционные алгоритмы}
\label{evo_algs}

\subsection{Основные принципы}
\textit{Эволюционные алгоритмы} применяются для решения задач оптимизации и используют механизмы, 
основанные на принципах природной эволюции.\cite{petrova_evo1} 

Эволюционные алгоритмы работают с множеством \textit{особей}-кандидатов на оптимальное решение. 
Как правило, особи представляются в виде точек в $K$-мерном пространстве, где $K$ - количество 
оптимизируемых критериев особи. Целью работы эволюционного алгоритма является нахождение наиболее 
приспособленной особи.

Функция, позволяющая оценить приспособленность особи, называется \textit{функцией приспособленности}.
В задачах многокритериальной оптимизации требуется рассматривать несколько функций приспособленности.

Каждая итерация эволюционного алгоритма работает с \textit{поколением} особей. Особи следующего 
поколения генерируются путем применения к особям текущего поколения операторов скрещивания, 
мутации и отбора.

В результате работы каждой итерации эволюционного алгоритма формируется множество особей следующего 
поколения, которое будет оптимизироваться следующей итерацией эволюционного алгоритма. Как правило, 
размер поколения фиксирован, поэтому в конце каждой итерации наименее приспособленные особи 
отбрасываются.

Эволюционный алгоритм завершает работу, как только достигнуто хотя бы одно из условий останова.
Как правило, в качестве критериев останова используется число итераций, либо значения функций
приспособленности.

\subsection{Популярные реализации}
\subsubsection{NSGA}
Алгоритм \textit{NSGA (Nondominated Sorting Genetic Algorithm)} является одним из первых алгоритмов
многокритериальной оптимизации. \cite{deb_nsga2}

Данный алгоритм на каждой итерации выполняет недоминирующую сортировку за $O(KN^3)$.
Помимо значений функции приспособленности, в качестве критерия перехода особи в следующую популяцию
используется \textit{фенотипическая дистанция}: если дистанция до какой-либо особи из следующего 
поколения меньше, чем задано параметром, то текущая особь в следующее поколение не добавляется. 
\cite{nsga1}

У данного алгоритма есть следующие недостатки: \cite{deb_nsga2}
\begin{enumerate}
\item Высокая вычислительная сложность недоминирующей сортировки;
\item Необходимость подбора параметра, определяющего минимально допустимое фенотипическое
	расстояние между особями;
\item Отсутствие элитизма.
\end{enumerate}

\subsubsection{Классический NSGA-II}
Алгоритм NSGA-II призван исправить основные недостатки алгоритма NSGA: \cite{deb_nsga2}
\begin{enumerate}
\item Сортировка выполняется за $O(KN^2)$;
\item При формировании нового поколения используются лучшие особи как из модифицированной
	(т. е. полученной в результате мутации и скрещивания особей текущего поколения) 
	популяции, так и из исходной;
\item При формировании нового поколения из комбинированной популяции отбрасываются особи
	с наибольшим рангом. Если в популяции остается только часть особей с некоторым рангом,
	выбираются особи с наибольшей \textit{crowding distance} - средней длиной ребра кубоида,
	образованного соседними особями (для крайних особей этот показатель равен $\infty$).
\end{enumerate}

Несмотря на это, у данного алгоритма есть недостаток, свойственный всем \textit{классическим}
(т. е. оперирующим поколениями особей) алгоритмам - невозможность эффективного распараллеливания.
Перед тем, как пересчитать ранг особи, алгоритм должен завершить расчет функций приспособленности 
всех особей. Решить данную проблему призваны \textit{инкрементальные (steady-state}) алгоритмы.
\cite{max_me_ss_nsga2}

\subsubsection{SPEA-II}
Алгоритм SPEA-II, хотя и позволяет достичь схожих с NSGA-II результатов, не использует процедуру 
недоминирующей сортировки (равно как и ранг точки) в том виде, в каком она используется в NSGA и 
NSGA-II и рассматривается в данной работе.

Для точки определены следующие величины: \cite{spea2}
\begin{enumerate}
 \item \textit{Сила} - число точек, над которыми текущая точка доминирует;
 \item \textit{Исходная жизнеспособность} (англ. \textit{raw fitness}) - сумма сил всех точек, 
	которые доминируют над текущей;
 \item \textit{Плотность} - величина, определяющая расстояние до ближайших точек.
	Данная величина по модулю меньше единицы.
\end{enumerate}

Особи оцениваются по сумме исходной жизнеспособности и плотности.

\subsubsection{Инкрементальные алгоритмы}
Наиболее очевидным способом реализовать инкрементальные версии алгоритмов NSGA-II и SPEA-II
является использование классических (\textit{population-based}) алгоритмов с модифицированной
популяцией размера 1. \cite{inc_nsga2_spea2}

Было отмечено, что инкрементальные версии алгоритмов позволяют получить сравнимый результат
за меньшее число итераций, ценой в 10-20 раз большего времени работы.

Основной причиной низкой производительности является скорость выполнения полной сортировки
множества особей. Лучшие из известных на сегодня алгоритмов позволяют выполнять сортировку 
за $O(Nlog^{K-1}N)$

\section{Алгоритмы недоминирующей сортировки}
\subsection{Алгоритмы полной сортировки}
TODO
\subsection{ENLU}
TODO