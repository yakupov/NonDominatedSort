\startprefacepage

Оптимизацией в математике называется задача нахождения экстремума (минимума или максимума) целевой функции в некоторой области конечномерного векторного пространства, ограниченной набором линейных и/или нелинейных равенств и/или неравенств \cite{wiki_opt}.

В процессе проектирования ставится обычно задача определения наилучших, в некотором смысле, структуры или значений параметров объектов. Такая задача называется оптимизационной. Если оптимизация связана с расчётом оптимальных значений параметров при заданной структуре объекта, то она называется параметрической оптимизацией. Задача выбора оптимальной структуры является структурной оптимизацией.

Стандартная математическая задача оптимизации формулируется таким образом. Среди элементов x, образующих множество X, найти такой элемент x*, который доставляет минимальное значение f(x*) заданной функции f(X).

Многокритериальная оптимизация или программирование — это процесс одновременной оптимизации двух или более конфликтующих целевых функций в заданной области определения.