\chapter{Результаты работы алгоритма}
\label{chapter3}

В данной главе проводится сравнение предложенного алгоритма с аналогами и оценка времени его работы 
на различных множествах тестовых данных.

\section{Конфигурация эксперимента}
\label{test_conf}
Было проведено сравнение с ENLU и недоминирующей сортировкой из NSGA-II.
Все алгоритмы тестировались на одном и том же, случайно сгенерированном множестве точек.
Инкрементальные алгоритмы тестировались путем последовательного добавления всех точек в структуру данных
и замера итогового времени работы всех вставок.
Для классической сортировки измерялось время сортировки всей популяции.

\subsection{Тестовые данные}
Использовались следующие виды входных данных:
\begin{itemize}
    \item ``квадрат'':   генерируются $N$ случайных точек из квадрата $N \times N$;
    \item ``параллель'': генерируются $N$ случайных точек, $N/2$ из которых лежат на прямой
                        $y = N - x$, остальные - на $y = N - x + 1$;
    \item ``diag1'':    генерируется последовательность из $N$ точек $(x, x)$,
                        начиная с максимального $x$;
    \item ``diag2'':    генерируется последовательность из $N$ точек $(x, x + 5)$ и $(x + 5, x)$ 
						одна за другой, начиная с максимального $x$;
    \item ``П'':   генерируется ``последовательно-перпендикулярная'' структура (напоминающая 
				   повернутую на 45 градусов букву \textsc{П}) следующего вида:
                        $N / 6$ точек на прямой $y = x + 5$,
                        $N / 6$ точек на прямой $y = x - 5$, 
                        $N / 3$ точек на прямой $y = N/3 - x - 4$, and
                        $N / 3$ точек на прямой $y = N/3 - x - 6$,
                   Пример такой структуры приведен на Рис.~\ref{parper-fig}.
\end{itemize}

\begin{figure}
\centering
\includegraphics[width=0.4\textwidth]{pic/scheme.3}
\caption{Пример структуры ``П'' при $N = 24$}\label{parper-fig}
\end{figure}

\section{Результаты тестирования}

Измерялось время работы и число сравнений.
Предложенный алгоритм показал значительно лучшие результаты чем сравниваемые решения.

\newpage

\newcommand\scalesize{0.77}
\newcommand\figalign\centering
%\newcommand\figalign\raggedleft

\begin{figure}
\figalign
\includegraphics[scale=\scalesize]{pic/plots.1}
\caption{Число сравнений для ``квадрата''}\label{plot-square-comp}
\end{figure}

\begin{figure}
\figalign
\includegraphics[scale=\scalesize]{pic/plots.3}
\caption{Число сравнений для ``параллели''}\label{plot-parallel-comp}
\end{figure}

\begin{figure}
\figalign
\includegraphics[scale=\scalesize]{pic/plots.5}
\caption{Число сравнений для ``diag1''}\label{plot-diag1-comp}
\end{figure}

\begin{figure}
\figalign
\includegraphics[scale=\scalesize]{pic/plots.7}
\caption{Число сравнений для ``diag2''}\label{plot-diag2-comp}
\end{figure}

\begin{figure}[b]
\figalign
\includegraphics[scale=\scalesize]{pic/plots.9}
\caption{Число сравнений для ``П''}\label{plot-parper-comp}
\end{figure}

\newpage

\begin{figure}
\figalign
\includegraphics[scale=\scalesize]{pic/plots.2}
\caption{Время работы для ``квадрата''}\label{plot-square-time}
\end{figure}

\begin{figure}
\figalign
\includegraphics[scale=\scalesize]{pic/plots.4}
\caption{Время работы для ``параллели''}\label{plot-parallel-time}
\end{figure}

\begin{figure}
\figalign
\includegraphics[scale=\scalesize]{pic/plots.6}
\caption{Время работы для ``diag1''}\label{plot-diag1-time}
\end{figure}

\begin{figure}
\figalign
\includegraphics[scale=\scalesize]{pic/plots.8}
\caption{Время работы для ``diag2''}\label{plot-diag2-time}
\end{figure}

\begin{figure}[b]
\figalign
\includegraphics[scale=\scalesize]{pic/plots.10}
\caption{Время работы для ``П''}\label{plot-parper-time}
\end{figure}
