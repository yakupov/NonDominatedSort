\documentclass{article}
\usepackage{amssymb}
\usepackage{amsthm}
\usepackage{amsmath}
\usepackage{latexsym}
\usepackage{algorithm}
\usepackage{algorithmicx}
\usepackage{algpseudocode}

\newtheorem{lemma1}{Lemma}
\newtheorem{th1}{Theorem}


\begin{document}

\section{Algorithm Description}\label{sec-algo}
In this section, the algorithm is described.
The population is being stored in an array of \textit{Ranks}.
Each \textit{Rank} contains set of \textit{Individuals} of a certain rank. 
This rank equals to the index of the corresponding \textit{Rank} (or
\textit{non-domination level}) in the population array.
In this algorithm \textit{Rank} is implemented with a Dekart tree.

\begin{algorithm}[H]
\begin{algorithmic}[1]
\Function{DetermineRank}{P, $p_n$}
	\State{$R_n \gets 0$}
	\For{$p \in P$} 
		\If{$x_p \le x_{p_n} \wedge y_p \le y_{p_n}$}
	        \State{$R_n \gets max(R_n, Rg(p) + 1)$}
	    \EndIf
	\EndFor
	\State\Return $R_n$
\EndFunction
\end{algorithmic}
\caption{The function \textsc{DetermineRank}. It calculates rank of the new
point $p_n$ basing the ranks of points from $p \in P$ who dominate $p_n$}
\label{DetermineRank}
\end{algorithm}

\begin{algorithm}[H]
\begin{algorithmic}[1]
\Procedure{AddPoint}{$P$, $p_n$}
	\If{$p_n \in P$} \Return
    \Else
    	\State{$R_n \gets \textsc{DetermineRank}(P, p_n$)}    
        \State{$i \gets 0$}
       	\State{$p_0 \gets p_n$} 
       	\State{$C_{-1} \gets \{p_n\}$}
       	\State{$C_0 \gets \{p: Rg(p) = R_n \wedge p_n \prec p\}$}
       	\While{$|C_i| \neq 0$}
       		\If{$\nexists P[R_n + i]$}
       			\State{$P[R_n + i] \gets C_i$}
       			\State\Return
       		\EndIf
      	 	\State{$P[R_n + i] \gets \textsc{CutTree}(P[R_n + i], C_i)$}
      	 	\State{$P[R_n + i] \gets \textsc{AddTree}(P[R_n + i], C_{i-1})$}
      	 	\State{$p_{i+1} \gets (\min{c \in C_i}x_c, \min{c \in C_i}y_c)$}
      	 	\State{$i \gets i + 1$}
      	 	\State{$C_i \gets \{p: Rg(p) = R_n + i \wedge p_i \prec p\}$}
        \EndWhile
        \State{$P[R_n + i] \gets \textsc{AddTree}(P[R_n + i], C_{i-1})$}
    \EndIf
\EndProcedure
\end{algorithmic}
\caption{The procedure \textsc{AddPoint}. On each step it splits tree of
current rank into two parts: points, that should change rank ($C_i$) and points
that should not. Then points, that have changed their rank on the previous steps,
are being added to the remainder. The proof is given in Theorem \ref{th1}}
\label{AddPoint}
\end{algorithm}


\section{Proof}
\begin{lemma1}
If:
\begin{equation} \label{eq:cdefrg} C = \{c: Rg(c) = R\}, \end{equation}
\begin{equation} \label{eq:p0def} p_0 = (min_{c \in C} c_x; min_{c \in C} c_y),
\end{equation}
\begin{equation} \label{eq:cdef} \nexists p': Rg(p') = R \quad \textrm{and}
\quad x_{p'} \in [min_{c \in C} c_x; max_{c \in C} c_x].
\end{equation}
Then:
\begin{equation}
\label{eq:rankd} D_{p_0} = \{d: p_0 \prec d \wedge Rg(d) > R\} =
D_C = \cup_{c \in C} \{d: c \prec d\}
\end{equation}
Where $D_e$ is a set of elements, dominated by $e$ (or at least by one member of
$e$, if this is a set).
\end{lemma1}

\begin{proof}
1. By definition of $p_0$,
$$ p_0 \preceq c, \forall c \in C $$
That leads, by the transitivity of $\preceq$, to the following:
$$ c \in C \prec d \Rightarrow p_0 \prec d $$
2. Let's assume the following:
\begin{align}
	\label{eq:p0dompd} p_0 \prec d, \\
	\label{eq:nexcdompd} \nexists c \in C: c \prec d.
\end{align}
\eqref{eq:p0dompd} can lead to the following cases:
\begin{enumerate}
\item $x_{d} = x_{p_0}$, $y_{d} > y_{p_0}$;

According to \eqref{eq:p0def}, $$\exists c_1 \in C: x_{c_1} = x_{p_0}$$
That means, that $x_{c_1} = x_{d}$. According to \eqref{eq:nexcdompd}, $y_{c_1}
>= y_{d}$.
That means, that 
$$
d \preceq c_1 \Rightarrow Rg(c_1) >= Rg(d)>^{\eqref{eq:rankd}} R
$$
This contradicts with \eqref{eq:cdefrg}.
\item $x_{d} > x_{p_0}$, $y_{d} = y_{p_0}$;

The proof is the same as for the previous case.
\item $x_{d} > x_{p_0}$, $y_{d} > y_{p_0}$;
$$
Rg(d) >^{\eqref{eq:rankd}} R \Rightarrow \exists p \notin C: Rg(p) = R, p \prec
d
$$
Let's introduce the following variables:
$$
c_1: c_1 \in C, x_{c_1} = x_{p_0}
$$
$$
c_2: c_2 \in C, y_{c_1} = y_{p_0}
$$
According to \eqref{eq:cdef}, 
$$
x_p < x_0 \vee y_p < y_0
$$
which means that $x_p < x_{c_1}$, and, according to the definition of rank, $y_p
> y_{c_1}$.
But $x_{d} > x_{p_0} = x_{c_1}$. Therefore:
$$ 
p \prec d \Rightarrow c_1 \prec d
$$
It contradicts with \eqref{eq:nexcdompd}.
The similar proof is applicable to $c_2$.
\end{enumerate}
\end{proof}

Let's define $Rg(p)$ as rank of $p$ before point addition, and $Rg'(p)$ as rank
of $p$ after point addition.
Let's define $R_i$:
$$ F_i: \{\forall f \in F_i: Rg(f) = F_i, p_i \not \prec f \}$$

\begin{th1}\label{th1}
Point $n$ was added. $Rg'(n) = R_0$.
The following statements are applicable for any iteration of point addition
algorithm:
\begin{enumerate}
  \item $ \forall i > 0 : R_i = R_0 + i;$
  \item $ \exists p_i, C_i: \{\forall c \in C_i: Rg(c) = R_i, p_i \prec c \};$
  \item $ Rg'(C_i) = R_i + 1;$
  \item $ Rg'(F_i) = R_i;$
  \item $ p_{i+1} = (min_{c \in C_i} c_x; min_{c \in C_i} c_y).$
\end{enumerate}
\end{th1}
\begin{proof}
The proof will be by induction.
\begin{enumerate}
  \item Base.
  \begin{enumerate}
  	\item $p_0 = n;$
  	\item $\exists C_0: \{\forall c \in C_0: Rg(c) = R_0, n \prec c \};$
  	\item $Rg'(C_0) = R_0 + 1 = R_i + 1;$
  	\item $Rg'(F_0)$ won't be changed;
  	\item $p_{i + 1} = (min_{c \in C_0} c_x; min_{c \in C_0} c_y).$
  \end{enumerate}
  \item Induction step.
  \begin{enumerate}
  	\item $ p_{i+1} = (min_{c \in C_i} c_x; min_{c \in C_i} c_y);$
  	\item $ Rg'(C_i) = R_i \Rightarrow^{lemma1} \forall {d: Rg(d) = R_i, p_i
  	\prec d: Rg'(d) = Rg(C_i) + 1 = R_i + 1}$;
  	\item $Rg'(F_i)$ won't be changed.
  \end{enumerate}
\end{enumerate}
\end{proof}

\section{Running time complexity}
\subsection{Running time of DetermineRank}
This function iterates over the whole population and compares the new point
against every one of the existing points. This gives us $N$ comparsions. Each
comparsion costs $O(k)$ operations, because it's required to compare $k$ components of each
individual.
Therefore, the running time complexity of DetermineRank is $O(N * k)$.
\subsection{Running time of AddPoint}
In the worst case, the new point will be added to the first non-domination
level. In this case we'll have to update all levels during the addition of this
point.
Let's introduce $L$ as the average size of a non-domination level and $M$ as the 
total number of non-domination levels.
The \textit{while} loop can give us up to $M$ iterations.
Obviously, \begin{equation} \label{eq:nlm}N \geq M * L \rightarrow M \leq N / L
\end{equation}
Inside each iteration:
\begin{enumerate}
  \item Procedure CutTree costs $O(\log(L))$ (according to the properties of
  Dekart tree);
  \item Procedure AddTree costs $O(\log(L))$ (according to the properties of
  Dekart tree);
  \item Calculation of $C_i$ costs $O(L * k)$ (because it's required to check
  all the points of the next rank).
\end{enumerate}
So, it's $M *(\log(L) + \log(L) + L * k) = 2 * M * \log(L) + M * L * k \leq 2 *
N * \log(L) / L + N * k$ in total.
This function reaches its maximum at $L = 2$.
Therefore, the running time complexity of AddPoint will be $O(N + N * k) =
O(N * k)$.
\subsection{Total running time}
So, the total running time of the proposed algorithm is $O(N * (k + 1) + N * k)
= O(N * k)$

\end{document}



