%%This is a very basic article template.
%%There is just one section and two subsections.
\documentclass{article}
\usepackage{amssymb}
\usepackage{amsthm}
\usepackage{amsmath}
\usepackage{latexsym}
\usepackage{algorithm}
\usepackage{algorithmicx}
\usepackage{algpseudocode}

\newtheorem{lemma1}{Lemma}
\newtheorem{th1}{Theorem}


\begin{document}

\section{Algorithm Description}\label{sec-algo}
In this section, the algorithm is described.
The population is being stored in an array of \textit{Ranks}.
Each \textit{Rank} contains set of \textit{Individuals} of a certain rank. 
This rank equals to the index of the corresponding \textit{Rank} in the
population array.

\begin{algorithm}[H]
\begin{algorithmic}[1]
\Function{DetermineRank}{P, $p_n$}
	\State{$R_n \gets 0$}
	\For{$p \in P$} 
		\If{$x_p \le x_{p_n} \wedge y_p \le y_{p_n}$}
	        \State{$R_n \gets max(R_n, Rg(p) + 1)$}
	    \EndIf
	\EndFor
	\State\Return $R_n$
\EndFunction
\end{algorithmic}
\caption{The function \textsc{DetermineRank}. It calculates rank of the new
point $p_n$ basing the ranks of points from $p \in P$ who dominate $p_n$}
\label{DetermineRank}
\end{algorithm}

\begin{algorithm}[H]
\begin{algorithmic}[1]
\Procedure{AddPoint}{$P$, $p_n$}
	\State{$R_n \gets \textsc{DetermineRank}(P, p_n$)}
	\If{$p_n \in P[R_n]$} \Return
    \Else
        \State{$i \gets 0$}
       	\State{$p_0 \gets p_n$} 
       	\State{$C_{-1} \gets \{p_n\}$}
       	\State{$C_0 \gets \{p: Rg(p) = R_n \wedge p_n \prec p\}$}
       	\While{$|C_i| \neq 0$}
      	 	\State{$P[R_n + i] \gets \textsc{CutTree}(P[R_n + i], C_i)$}
      	 	\State{$P[R_n + i] \gets \textsc{AddTree}(P[R_n + i], C_{i-1})$}
      	 	\State{$p_{i+1} \gets (\min{c \in C_i}x_c, \min{c \in C_i}y_c)$}
      	 	\State{$i \gets i + 1$}
      	 	\State{$C_i \gets \{p: Rg(p) = R_n + i \wedge p_i \prec p\}$}
        \EndWhile
        \State{$P[R_n + i] \gets \textsc{AddTree}(P[R_n + i], C_{i-1})$}
    \EndIf
\EndProcedure
\end{algorithmic}
\caption{The procedure \textsc{AddPoint}. On each step it splits tree of
current rank into two parts: points, that should change rank ($C_i$) and points
that should not. Then points, that have changed their rank on the previous steps,
are being added to the remainder. The proof is given in Theorem \ref{th1}}
\label{AddPoint}
\end{algorithm}


\section{Proof}
\begin{lemma1}
If:
\begin{equation} \label{eq:cdefrg} C = \{c: Rg(c) = R\}, \end{equation}
\begin{equation} \label{eq:p0def} p_0 = (min_{c \in C} c_x; min_{c \in C} c_y),
\end{equation}
\begin{equation} \label{eq:cdef} \nexists p': Rg(p') = R \quad \textrm{and}
\quad x_{p'} \in [min_{c \in C} c_x; max_{c \in C} c_x].
\end{equation}
Then:
\begin{equation}
\label{eq:rankd} D_{p_0} = \{d: p_0 \prec d \wedge Rg(d) > R\} =
D_C = \cup_{c \in C} \{d: c \prec d\}
\end{equation}
Where $D_e$ is a set of elements, dominated by $e$ (or by any member of
$e$, if this is a set).
\end{lemma1}

\begin{proof}
1. By definition of $p_0$,
$$ p_0 \preceq c, \forall c \in C $$
That leads, by the transitivity of $\preceq$, to the following:
$$ c \in C \prec d \Rightarrow p_0 \prec d $$
2. Let's suggest the following:
\begin{align}
	\label{eq:p0dompd} p_0 \prec d, \\
	\label{eq:nexcdompd} \nexists c \in C: c \prec d.
\end{align}
\eqref{eq:p0dompd} can lead to the following cases:
\begin{enumerate}
\item $x_{d} = x_{p_0}$, $y_{d} > y_{p_0}$;

According to \eqref{eq:p0def}, $$\exists c_1 \in C: x_{c_1} = x_{p_0}$$
That means, that $x_{c_1} = x_{d}$. According to \eqref{eq:nexcdompd}, $y_{c_1}
>= y_{d}$.
That means, that 
$$
d \preceq c_1 \Rightarrow Rg(c_1) >= Rg(d)>^{\eqref{eq:rankd}} R
$$
This contradicts with \eqref{eq:cdefrg}.
\item $x_{d} > x_{p_0}$, $y_{d} = y_{p_0}$;

The proof is the same as for the previous case.
\item $x_{d} > x_{p_0}$, $y_{d} > y_{p_0}$;
$$
Rg(d) >^{\eqref{eq:rankd}} R \Rightarrow \exists p \notin C: Rg(p) = R, p \prec
d
$$
Let's state the following:
$$
c_1: c_1 \in C, x_{c_1} = x_{p_0}
$$
$$
c_2: c_2 \in C, y_{c_1} = y_{p_0}
$$
According to \eqref{eq:cdef}, 
$$
x_p < x_0 \vee y_p < y_0
$$
which means that $x_p < x_{c_1}$, and, according to the definition of rank, $y_p
> y_{c_1}$.
But $x_{d} > x_{p_0} = x_{c_1}$. Therefore:
$$ 
p \prec d \Rightarrow c_1 \prec d
$$
It contradicts with \eqref{eq:nexcdompd}.
The similar proof is applicable to $c_2$.
\end{enumerate}
\end{proof}

Let's define $Rg(p)$ as rank of $p$ before point addition, and $Rg'(p)$ as rank
of $p$ after point addition.
Let's define $R_i$:
$$ F_i: \{\forall f \in F_i: Rg(f) = F_i, p_i \not \prec f \}$$

\begin{th1}\label{th1}
Point $n$ was added. $Rg'(n) = R_0$.
The following statements are applicable for any iteration of point addition
algorithm:
\begin{enumerate}
  \item $ \forall i > 0 : R_i = R_0 + i;$
  \item $ \exists p_i, C_i: \{\forall c \in C_i: Rg(c) = R_i, p_i \prec c \};$
  \item $ Rg'(C_i) = R_i + 1;$
  \item $ Rg'(F_i) = R_i;$
  \item $ p_{i+1} = (min_{c \in C_i} c_x; min_{c \in C_i} c_y).$
\end{enumerate}
\end{th1}
\begin{proof}
The proof will be by induction.
\begin{enumerate}
  \item Base.
  \begin{enumerate}
  	\item $p_0 = n;$
  	\item $\exists C_0: \{\forall c \in C_0: Rg(c) = R_0, n \prec c \};$
  	\item $Rg'(C_0) = R_0 + 1 = R_i + 1;$
  	\item $Rg'(F_0)$ won't be changed;
  	\item $p_{i + 1} = (min_{c \in C_0} c_x; min_{c \in C_0} c_y).$
  \end{enumerate}
  \item Induction step.
  \begin{enumerate}
  	\item $ p_{i+1} = (min_{c \in C_i} c_x; min_{c \in C_i} c_y);$
  	\item $ Rg'(C_i) = R_i \Rightarrow^{lemma1} \forall {d: Rg(d) = R_i, p_i
  	\prec d: Rg'(d) = Rg(C_i) + 1 = R_i + 1}$;
  	\item $Rg'(F_i)$ won't be changed.
  \end{enumerate}
\end{enumerate}
\end{proof}


\end{document}



